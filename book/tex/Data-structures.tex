\chapter{Data structures}\label{data-structures}

This chapter summarises the most important data structures in base R.
You've probably used many (if not all) of them before, but you may not
have thought deeply about how they are interrelated. In this brief
overview, I won't discuss individual types in depth. Instead, I'll show
you how they fit together as a whole. If you need more details, you can
find them in R's documentation.

R's base data structures can be organised by their dimensionality (1d,
2d, or nd) and whether they're homogeneous (all contents must be of the
same type) or heterogeneous (the contents can be of different types).
This gives rise to the five data types most often used in data analysis:

\begin{longtable}[c]{@{}lll@{}}
\toprule
& Homogeneous & Heterogeneous\tabularnewline
\midrule
\endhead
1d & Atomic vector & List\tabularnewline
2d & Matrix & Data frame\tabularnewline
nd & Array &\tabularnewline
\bottomrule
\end{longtable}

Almost all other objects are built upon these foundations. In
\hyperref[oo]{the OO field guide} you'll see how more complicated
objects are built of these simple pieces. Note that R has no
0-dimensional, or scalar types. Individual numbers or strings, which you
might think would be scalars, are actually vectors of length one.

Given an object, the best way to understand what data structures it's
composed of is to use \texttt{str()}. \texttt{str()} is short for
structure and it gives a compact, human readable description of any R
data structure. \indexc{str()}

\paragraph{Quiz}

Take this short quiz to determine if you need to read this chapter. If
the answers quickly come to mind, you can comfortably skip this chapter.
You can check your answers in
\hyperref[data-structure-answers]{answers}.

\begin{enumerate}
\def\labelenumi{\arabic{enumi}.}
\item
  What are the three properties of a vector, other than its contents?
\item
  What are the four common types of atomic vectors? What are the two
  rare types?
\item
  What are attributes? How do you get them and set them?
\item
  How is a list different from an atomic vector? How is a matrix
  different from a data frame?
\item
  Can you have a list that is a matrix? Can a data frame have a column
  that is a matrix?
\end{enumerate}

\paragraph{Outline}

\begin{itemize}
\item
  \hyperref[vectors]{Vectors} introduces you to atomic vectors and
  lists, R's 1d data structures.
\item
  \hyperref[attributes]{Attributes} takes a small detour to discuss
  attributes, R's flexible metadata specification. Here you'll learn
  about factors, an important data structure created by setting
  attributes of an atomic vector.
\item
  \hyperref[matrices-and-arrays]{Matrices and arrays} introduces
  matrices and arrays, data structures for storing 2d and higher
  dimensional data.
\item
  \hyperref[data-frames]{Data frames} teaches you about the data frame,
  the most important data structure for storing data in R. Data frames
  combine the behaviour of lists and matrices to make a structure
  ideally suited for the needs of statistical data.
\end{itemize}

\hyperdef{}{vectors}{\section{Vectors}\label{vectors}}

The basic data structure in R is the vector. Vectors come in two
flavours: atomic vectors and lists. They have three common properties:

\begin{itemize}
\itemsep1pt\parskip0pt\parsep0pt
\item
  Type, \texttt{typeof()}, what it is.
\item
  Length, \texttt{length()}, how many elements it contains.
\item
  Attributes, \texttt{attributes()}, additional arbitrary metadata.
\end{itemize}

They differ in the types of their elements: all elements of an atomic
vector must be the same type, whereas the elements of a list can have
different types.

NB: \texttt{is.vector()} does not test if an object is a vector. Instead
it returns \texttt{TRUE} only if the object is a vector with no
attributes apart from names. Use
\texttt{is.atomic(x) \textbar{}\textbar{} is.list(x)} to test if an
object is actually a vector.

\subsection{Atomic vectors}

There are four common types of atomic vectors that I'll discuss in
detail: logical, integer, double (often called numeric), and character.
There are two rare types that I will not discuss further: complex and
raw. \index{atomic vectors} \index{vectors!atomic|see{atomic vectors}}

Atomic vectors are usually created with \texttt{c()}, short for combine:
\indexc{c()}

\begin{Shaded}
\begin{Highlighting}[]
\NormalTok{dbl_var <-}\StringTok{ }\KeywordTok{c}\NormalTok{(}\DecValTok{1}\NormalTok{, }\FloatTok{2.5}\NormalTok{, }\FloatTok{4.5}\NormalTok{)}
\CommentTok{# With the L suffix, you get an integer rather than a double}
\NormalTok{int_var <-}\StringTok{ }\KeywordTok{c}\NormalTok{(1L, 6L, 10L)}
\CommentTok{# Use TRUE and FALSE (or T and F) to create logical vectors}
\NormalTok{log_var <-}\StringTok{ }\KeywordTok{c}\NormalTok{(}\OtherTok{TRUE}\NormalTok{, }\OtherTok{FALSE}\NormalTok{, T, F)}
\NormalTok{chr_var <-}\StringTok{ }\KeywordTok{c}\NormalTok{(}\StringTok{"these are"}\NormalTok{, }\StringTok{"some strings"}\NormalTok{)}
\end{Highlighting}
\end{Shaded}

Atomic vectors are always flat, even if you nest \texttt{c()}'s:

\begin{Shaded}
\begin{Highlighting}[]
\KeywordTok{c}\NormalTok{(}\DecValTok{1}\NormalTok{, }\KeywordTok{c}\NormalTok{(}\DecValTok{2}\NormalTok{, }\KeywordTok{c}\NormalTok{(}\DecValTok{3}\NormalTok{, }\DecValTok{4}\NormalTok{)))}
\CommentTok{#> [1] 1 2 3 4}
\CommentTok{# the same as}
\KeywordTok{c}\NormalTok{(}\DecValTok{1}\NormalTok{, }\DecValTok{2}\NormalTok{, }\DecValTok{3}\NormalTok{, }\DecValTok{4}\NormalTok{)}
\CommentTok{#> [1] 1 2 3 4}
\end{Highlighting}
\end{Shaded}

Missing values are specified with \texttt{NA}, which is a logical vector
of length 1. \texttt{NA} will always be coerced to the correct type if
used inside \texttt{c()}, or you can create \texttt{NA}s of a specific
type with \texttt{NA\_real\_} (a double vector), \texttt{NA\_integer\_}
and \texttt{NA\_character\_}. \indexc{NA}

\subsubsection{Types and tests}

Given a vector, you can determine its type with \texttt{typeof()}, or
check if it's a specific type with an ``is'' function:
\texttt{is.character()}, \texttt{is.double()}, \texttt{is.integer()},
\texttt{is.logical()}, or, more generally, \texttt{is.atomic()}.
\indexc{typeof()}

\begin{Shaded}
\begin{Highlighting}[]
\NormalTok{int_var <-}\StringTok{ }\KeywordTok{c}\NormalTok{(1L, 6L, 10L)}
\KeywordTok{typeof}\NormalTok{(int_var)}
\CommentTok{#> [1] "integer"}
\KeywordTok{is.integer}\NormalTok{(int_var)}
\CommentTok{#> [1] TRUE}
\KeywordTok{is.atomic}\NormalTok{(int_var)}
\CommentTok{#> [1] TRUE}

\NormalTok{dbl_var <-}\StringTok{ }\KeywordTok{c}\NormalTok{(}\DecValTok{1}\NormalTok{, }\FloatTok{2.5}\NormalTok{, }\FloatTok{4.5}\NormalTok{)}
\KeywordTok{typeof}\NormalTok{(dbl_var)}
\CommentTok{#> [1] "double"}
\KeywordTok{is.double}\NormalTok{(dbl_var)}
\CommentTok{#> [1] TRUE}
\KeywordTok{is.atomic}\NormalTok{(dbl_var)}
\CommentTok{#> [1] TRUE}
\end{Highlighting}
\end{Shaded}

NB: \texttt{is.numeric()} is a general test for the ``numberliness'' of
a vector and returns \texttt{TRUE} for both integer and double vectors.
It is not a specific test for double vectors, which are often called
numeric. \indexc{is.numeric()}

\begin{Shaded}
\begin{Highlighting}[]
\KeywordTok{is.numeric}\NormalTok{(int_var)}
\CommentTok{#> [1] TRUE}
\KeywordTok{is.numeric}\NormalTok{(dbl_var)}
\CommentTok{#> [1] TRUE}
\end{Highlighting}
\end{Shaded}

\subsubsection{Coercion}

All elements of an atomic vector must be the same type, so when you
attempt to combine different types they will be \textbf{coerced} to the
most flexible type. Types from least to most flexible are: logical,
integer, double, and character. \index{coercion}

For example, combining a character and an integer yields a character:

\begin{Shaded}
\begin{Highlighting}[]
\KeywordTok{str}\NormalTok{(}\KeywordTok{c}\NormalTok{(}\StringTok{"a"}\NormalTok{, }\DecValTok{1}\NormalTok{))}
\CommentTok{#>  chr [1:2] "a" "1"}
\end{Highlighting}
\end{Shaded}

When a logical vector is coerced to an integer or double, \texttt{TRUE}
becomes 1 and \texttt{FALSE} becomes 0. This is very useful in
conjunction with \texttt{sum()} and \texttt{mean()}

\begin{Shaded}
\begin{Highlighting}[]
\NormalTok{x <-}\StringTok{ }\KeywordTok{c}\NormalTok{(}\OtherTok{FALSE}\NormalTok{, }\OtherTok{FALSE}\NormalTok{, }\OtherTok{TRUE}\NormalTok{)}
\KeywordTok{as.numeric}\NormalTok{(x)}
\CommentTok{#> [1] 0 0 1}

\CommentTok{# Total number of TRUEs}
\KeywordTok{sum}\NormalTok{(x)}
\CommentTok{#> [1] 1}

\CommentTok{# Proportion that are TRUE}
\KeywordTok{mean}\NormalTok{(x)}
\CommentTok{#> [1] 0.333}
\end{Highlighting}
\end{Shaded}

Coercion often happens automatically. Most mathematical functions
(\texttt{+}, \texttt{log}, \texttt{abs}, etc.) will coerce to a double
or integer, and most logical operations (\texttt{\&},
\texttt{\textbar{}}, \texttt{any}, etc) will coerce to a logical. You
will usually get a warning message if the coercion might lose
information. If confusion is likely, explicitly coerce with
\texttt{as.character()}, \texttt{as.double()}, \texttt{as.integer()}, or
\texttt{as.logical()}.

\subsection{Lists}

Lists are different from atomic vectors because their elements can be of
any type, including lists. You construct lists by using \texttt{list()}
instead of \texttt{c()}: \index{lists} \index{vectors!lists|see{lists}}

\begin{Shaded}
\begin{Highlighting}[]
\NormalTok{x <-}\StringTok{ }\KeywordTok{list}\NormalTok{(}\DecValTok{1}\NormalTok{:}\DecValTok{3}\NormalTok{, }\StringTok{"a"}\NormalTok{, }\KeywordTok{c}\NormalTok{(}\OtherTok{TRUE}\NormalTok{, }\OtherTok{FALSE}\NormalTok{, }\OtherTok{TRUE}\NormalTok{), }\KeywordTok{c}\NormalTok{(}\FloatTok{2.3}\NormalTok{, }\FloatTok{5.9}\NormalTok{))}
\KeywordTok{str}\NormalTok{(x)}
\CommentTok{#> List of 4}
\CommentTok{#>  $ : int [1:3] 1 2 3}
\CommentTok{#>  $ : chr "a"}
\CommentTok{#>  $ : logi [1:3] TRUE FALSE TRUE}
\CommentTok{#>  $ : num [1:2] 2.3 5.9}
\end{Highlighting}
\end{Shaded}

Lists are sometimes called \textbf{recursive} vectors, because a list
can contain other lists. This makes them fundamentally different from
atomic vectors.

\begin{Shaded}
\begin{Highlighting}[]
\NormalTok{x <-}\StringTok{ }\KeywordTok{list}\NormalTok{(}\KeywordTok{list}\NormalTok{(}\KeywordTok{list}\NormalTok{(}\KeywordTok{list}\NormalTok{())))}
\KeywordTok{str}\NormalTok{(x)}
\CommentTok{#> List of 1}
\CommentTok{#>  $ :List of 1}
\CommentTok{#>   ..$ :List of 1}
\CommentTok{#>   .. ..$ : list()}
\KeywordTok{is.recursive}\NormalTok{(x)}
\CommentTok{#> [1] TRUE}
\end{Highlighting}
\end{Shaded}

\texttt{c()} will combine several lists into one. If given a combination
of atomic vectors and lists, \texttt{c()} will coerce the vectors to
lists before combining them. Compare the results of \texttt{list()} and
\texttt{c()}:

\begin{Shaded}
\begin{Highlighting}[]
\NormalTok{x <-}\StringTok{ }\KeywordTok{list}\NormalTok{(}\KeywordTok{list}\NormalTok{(}\DecValTok{1}\NormalTok{, }\DecValTok{2}\NormalTok{), }\KeywordTok{c}\NormalTok{(}\DecValTok{3}\NormalTok{, }\DecValTok{4}\NormalTok{))}
\NormalTok{y <-}\StringTok{ }\KeywordTok{c}\NormalTok{(}\KeywordTok{list}\NormalTok{(}\DecValTok{1}\NormalTok{, }\DecValTok{2}\NormalTok{), }\KeywordTok{c}\NormalTok{(}\DecValTok{3}\NormalTok{, }\DecValTok{4}\NormalTok{))}
\KeywordTok{str}\NormalTok{(x)}
\CommentTok{#> List of 2}
\CommentTok{#>  $ :List of 2}
\CommentTok{#>   ..$ : num 1}
\CommentTok{#>   ..$ : num 2}
\CommentTok{#>  $ : num [1:2] 3 4}
\KeywordTok{str}\NormalTok{(y)}
\CommentTok{#> List of 4}
\CommentTok{#>  $ : num 1}
\CommentTok{#>  $ : num 2}
\CommentTok{#>  $ : num 3}
\CommentTok{#>  $ : num 4}
\end{Highlighting}
\end{Shaded}

The \texttt{typeof()} a list is \texttt{list}. You can test for a list
with \texttt{is.list()} and coerce to a list with \texttt{as.list()}.
You can turn a list into an atomic vector with \texttt{unlist()}. If the
elements of a list have different types, \texttt{unlist()} uses the same
coercion rules as \texttt{c()}.

Lists are used to build up many of the more complicated data structures
in R. For example, both data frames (described in
\hyperref[data-frames]{data frames}) and linear models objects (as
produced by \texttt{lm()}) are lists:

\begin{Shaded}
\begin{Highlighting}[]
\KeywordTok{is.list}\NormalTok{(mtcars)}
\CommentTok{#> [1] TRUE}

\NormalTok{mod <-}\StringTok{ }\KeywordTok{lm}\NormalTok{(mpg ~}\StringTok{ }\NormalTok{wt, }\DataTypeTok{data =} \NormalTok{mtcars)}
\KeywordTok{is.list}\NormalTok{(mod)}
\CommentTok{#> [1] TRUE}
\end{Highlighting}
\end{Shaded}

\subsection{Exercises}

\begin{enumerate}
\def\labelenumi{\arabic{enumi}.}
\item
  What are the six types of atomic vector? How does a list differ from
  an atomic vector?
\item
  What makes \texttt{is.vector()} and \texttt{is.numeric()}
  fundamentally different to \texttt{is.list()} and
  \texttt{is.character()}?
\item
  Test your knowledge of vector coercion rules by predicting the output
  of the following uses of \texttt{c()}:

\begin{Shaded}
\begin{Highlighting}[]
\KeywordTok{c}\NormalTok{(}\DecValTok{1}\NormalTok{, }\OtherTok{FALSE}\NormalTok{)}
\KeywordTok{c}\NormalTok{(}\StringTok{"a"}\NormalTok{, }\DecValTok{1}\NormalTok{)}
\KeywordTok{c}\NormalTok{(}\KeywordTok{list}\NormalTok{(}\DecValTok{1}\NormalTok{), }\StringTok{"a"}\NormalTok{)}
\KeywordTok{c}\NormalTok{(}\OtherTok{TRUE}\NormalTok{, 1L)}
\end{Highlighting}
\end{Shaded}
\item
  Why do you need to use \texttt{unlist()} to convert a list to an
  atomic vector? Why doesn't \texttt{as.vector()} work?
\item
  Why is \texttt{1 == "1"} true? Why is \texttt{-1 \textless{} FALSE}
  true? Why is \texttt{"one" \textless{} 2} false?
\item
  Why is the default missing value, \texttt{NA}, a logical vector?
  What's special about logical vectors? (Hint: think about
  \texttt{c(FALSE, NA\_character\_)}.)
\end{enumerate}

\hyperdef{}{attributes}{\section{Attributes}\label{attributes}}

All objects can have arbitrary additional attributes, used to store
metadata about the object. Attributes can be thought of as a named list
(with unique names). Attributes can be accessed individually with
\texttt{attr()} or all at once (as a list) with \texttt{attributes()}.
\index{attributes}

\begin{Shaded}
\begin{Highlighting}[]
\NormalTok{y <-}\StringTok{ }\DecValTok{1}\NormalTok{:}\DecValTok{10}
\KeywordTok{attr}\NormalTok{(y, }\StringTok{"my_attribute"}\NormalTok{) <-}\StringTok{ "This is a vector"}
\KeywordTok{attr}\NormalTok{(y, }\StringTok{"my_attribute"}\NormalTok{)}
\CommentTok{#> [1] "This is a vector"}
\KeywordTok{str}\NormalTok{(}\KeywordTok{attributes}\NormalTok{(y))}
\CommentTok{#> List of 1}
\CommentTok{#>  $ my_attribute: chr "This is a vector"}
\end{Highlighting}
\end{Shaded}

The \texttt{structure()} function returns a new object with modified
attributes: \indexc{structure()}

\begin{Shaded}
\begin{Highlighting}[]
\KeywordTok{structure}\NormalTok{(}\DecValTok{1}\NormalTok{:}\DecValTok{10}\NormalTok{, }\DataTypeTok{my_attribute =} \StringTok{"This is a vector"}\NormalTok{)}
\CommentTok{#>  [1]  1  2  3  4  5  6  7  8  9 10}
\CommentTok{#> attr(,"my_attribute")}
\CommentTok{#> [1] "This is a vector"}
\end{Highlighting}
\end{Shaded}

By default, most attributes are lost when modifying a vector:

\begin{Shaded}
\begin{Highlighting}[]
\KeywordTok{attributes}\NormalTok{(y[}\DecValTok{1}\NormalTok{])}
\CommentTok{#> NULL}
\KeywordTok{attributes}\NormalTok{(}\KeywordTok{sum}\NormalTok{(y))}
\CommentTok{#> NULL}
\end{Highlighting}
\end{Shaded}

The only attributes not lost are the three most important:

\begin{itemize}
\item
  Names, a character vector giving each element a name, described in
  \hyperref[vector-names]{names}.
\item
  Dimensions, used to turn vectors into matrices and arrays, described
  in \hyperref[matrices-and-arrays]{matrices and arrays}.
\item
  Class, used to implement the S3 object system, described in
  \hyperref[s3]{S3}.
\end{itemize}

Each of these attributes has a specific accessor function to get and set
values. When working with these attributes, use \texttt{names(x)},
\texttt{dim(x)}, and \texttt{class(x)}, not \texttt{attr(x, "names")},
\texttt{attr(x, "dim")}, and \texttt{attr(x, "class")}.

\hyperdef{}{vector-names}{\subsubsection{Names}\label{vector-names}}

You can name a vector in three ways: \index{attributes|names}

\begin{itemize}
\item
  When creating it: \texttt{x \textless{}- c(a = 1, b = 2, c = 3)}.
\item
  By modifying an existing vector in place:
  \texttt{x \textless{}- 1:3; names(x) \textless{}- c("a", "b", "c")}.
  \indexc{names()}
\item
  By creating a modified copy of a vector:
  \texttt{x \textless{}- setNames(1:3, c("a", "b", "c"))}.
  \indexc{setNames()}
\end{itemize}

Names don't have to be unique. However, character subsetting, described
in \hyperref[lookup-tables]{subsetting}, is the most important reason to
use names and it is most useful when the names are unique.

Not all elements of a vector need to have a name. If some names are
missing, \texttt{names()} will return an empty string for those
elements. If all names are missing, \texttt{names()} will return
\texttt{NULL}.

\begin{Shaded}
\begin{Highlighting}[]
\NormalTok{y <-}\StringTok{ }\KeywordTok{c}\NormalTok{(}\DataTypeTok{a =} \DecValTok{1}\NormalTok{, }\DecValTok{2}\NormalTok{, }\DecValTok{3}\NormalTok{)}
\KeywordTok{names}\NormalTok{(y)}
\CommentTok{#> [1] "a" ""  ""}

\NormalTok{z <-}\StringTok{ }\KeywordTok{c}\NormalTok{(}\DecValTok{1}\NormalTok{, }\DecValTok{2}\NormalTok{, }\DecValTok{3}\NormalTok{)}
\KeywordTok{names}\NormalTok{(z)}
\CommentTok{#> NULL}
\end{Highlighting}
\end{Shaded}

You can create a new vector without names using \texttt{unname(x)}, or
remove names in place with \texttt{names(x) \textless{}- NULL}.

\subsection{Factors}

One important use of attributes is to define factors. A factor is a
vector that can contain only predefined values, and is used to store
categorical data. Factors are built on top of integer vectors using two
attributes: the \texttt{class()}, ``factor'', which makes them behave
differently from regular integer vectors, and the \texttt{levels()},
which defines the set of allowed values. \index{factors|(}

\begin{Shaded}
\begin{Highlighting}[]
\NormalTok{x <-}\StringTok{ }\KeywordTok{factor}\NormalTok{(}\KeywordTok{c}\NormalTok{(}\StringTok{"a"}\NormalTok{, }\StringTok{"b"}\NormalTok{, }\StringTok{"b"}\NormalTok{, }\StringTok{"a"}\NormalTok{))}
\NormalTok{x}
\CommentTok{#> [1] a b b a}
\CommentTok{#> Levels: a b}
\KeywordTok{class}\NormalTok{(x)}
\CommentTok{#> [1] "factor"}
\KeywordTok{levels}\NormalTok{(x)}
\CommentTok{#> [1] "a" "b"}

\CommentTok{# You can't use values that are not in the levels}
\NormalTok{x[}\DecValTok{2}\NormalTok{] <-}\StringTok{ "c"}
\CommentTok{#> Warning in `[<-.factor`(`*tmp*`, 2, value = "c"): invalid}
\CommentTok{#> factor level, NA generated}
\NormalTok{x}
\CommentTok{#> [1] a    <NA> b    a   }
\CommentTok{#> Levels: a b}

\CommentTok{# NB: you can't combine factors}
\KeywordTok{c}\NormalTok{(}\KeywordTok{factor}\NormalTok{(}\StringTok{"a"}\NormalTok{), }\KeywordTok{factor}\NormalTok{(}\StringTok{"b"}\NormalTok{))}
\CommentTok{#> [1] 1 1}
\end{Highlighting}
\end{Shaded}

Factors are useful when you know the possible values a variable may
take, even if you don't see all values in a given dataset. Using a
factor instead of a character vector makes it obvious when some groups
contain no observations:

\begin{Shaded}
\begin{Highlighting}[]
\NormalTok{sex_char <-}\StringTok{ }\KeywordTok{c}\NormalTok{(}\StringTok{"m"}\NormalTok{, }\StringTok{"m"}\NormalTok{, }\StringTok{"m"}\NormalTok{)}
\NormalTok{sex_factor <-}\StringTok{ }\KeywordTok{factor}\NormalTok{(sex_char, }\DataTypeTok{levels =} \KeywordTok{c}\NormalTok{(}\StringTok{"m"}\NormalTok{, }\StringTok{"f"}\NormalTok{))}

\KeywordTok{table}\NormalTok{(sex_char)}
\CommentTok{#> sex_char}
\CommentTok{#> m }
\CommentTok{#> 3}
\KeywordTok{table}\NormalTok{(sex_factor)}
\CommentTok{#> sex_factor}
\CommentTok{#> m f }
\CommentTok{#> 3 0}
\end{Highlighting}
\end{Shaded}

Sometimes when a data frame is read directly from a file, a column you'd
thought would produce a numeric vector instead produces a factor. This
is caused by a non-numeric value in the column, often a missing value
encoded in a special way like \texttt{.} or \texttt{-}. To remedy the
situation, coerce the vector from a factor to a character vector, and
then from a character to a double vector. (Be sure to check for missing
values after this process.) Of course, a much better plan is to discover
what caused the problem in the first place and fix that; using the
\texttt{na.strings} argument to \texttt{read.csv()} is often a good
place to start.

\begin{Shaded}
\begin{Highlighting}[]
\CommentTok{# Reading in "text" instead of from a file here:}
\NormalTok{z <-}\StringTok{ }\KeywordTok{read.csv}\NormalTok{(}\DataTypeTok{text =} \StringTok{"value}\CharTok{\textbackslash{}n}\StringTok{12}\CharTok{\textbackslash{}n}\StringTok{1}\CharTok{\textbackslash{}n}\StringTok{.}\CharTok{\textbackslash{}n}\StringTok{9"}\NormalTok{)}
\KeywordTok{typeof}\NormalTok{(z$value)}
\CommentTok{#> [1] "integer"}
\KeywordTok{as.double}\NormalTok{(z$value)}
\CommentTok{#> [1] 3 2 1 4}
\CommentTok{# Oops, that's not right: 3 2 1 4 are the levels of a factor, }
\CommentTok{# not the values we read in!}
\KeywordTok{class}\NormalTok{(z$value)}
\CommentTok{#> [1] "factor"}
\CommentTok{# We can fix it now:}
\KeywordTok{as.double}\NormalTok{(}\KeywordTok{as.character}\NormalTok{(z$value))}
\CommentTok{#> Warning: NAs introduced by coercion}
\CommentTok{#> [1] 12  1 NA  9}
\CommentTok{# Or change how we read it in:}
\NormalTok{z <-}\StringTok{ }\KeywordTok{read.csv}\NormalTok{(}\DataTypeTok{text =} \StringTok{"value}\CharTok{\textbackslash{}n}\StringTok{12}\CharTok{\textbackslash{}n}\StringTok{1}\CharTok{\textbackslash{}n}\StringTok{.}\CharTok{\textbackslash{}n}\StringTok{9"}\NormalTok{, }\DataTypeTok{na.strings=}\StringTok{"."}\NormalTok{)}
\KeywordTok{typeof}\NormalTok{(z$value)}
\CommentTok{#> [1] "integer"}
\KeywordTok{class}\NormalTok{(z$value)}
\CommentTok{#> [1] "integer"}
\NormalTok{z$value}
\CommentTok{#> [1] 12  1 NA  9}
\CommentTok{# Perfect! :)}
\end{Highlighting}
\end{Shaded}

Unfortunately, most data loading functions in R automatically convert
character vectors to factors. This is suboptimal, because there's no way
for those functions to know the set of all possible levels or their
optimal order. Instead, use the argument
\texttt{stringsAsFactors = FALSE} to suppress this behaviour, and then
manually convert character vectors to factors using your knowledge of
the data. A global option, \texttt{options(stringsAsFactors = FALSE)},
is available to control this behaviour, but I don't recommend using it.
Changing a global option may have unexpected consequences when combined
with other code (either from packages, or code that you're
\texttt{source()}ing), and global options make code harder to understand
because they increase the number of lines you need to read to understand
how a single line of code will behave. \indexc{stringsAsFactors}

While factors look (and often behave) like character vectors, they are
actually integers. Be careful when treating them like strings. Some
string methods (like \texttt{gsub()} and \texttt{grepl()}) will coerce
factors to strings, while others (like \texttt{nchar()}) will throw an
error, and still others (like \texttt{c()}) will use the underlying
integer values. For this reason, it's usually best to explicitly convert
factors to character vectors if you need string-like behaviour. In early
versions of R, there was a memory advantage to using factors instead of
character vectors, but this is no longer the case. \index{factors|)}

\subsection{Exercises}

\begin{enumerate}
\def\labelenumi{\arabic{enumi}.}
\item
  An early draft used this code to illustrate \texttt{structure()}:

\begin{Shaded}
\begin{Highlighting}[]
\KeywordTok{structure}\NormalTok{(}\DecValTok{1}\NormalTok{:}\DecValTok{5}\NormalTok{, }\DataTypeTok{comment =} \StringTok{"my attribute"}\NormalTok{)}
\CommentTok{#> [1] 1 2 3 4 5}
\end{Highlighting}
\end{Shaded}

  But when you print that object you don't see the comment attribute.
  Why? Is the attribute missing, or is there something else special
  about it? (Hint: try using help.) \index{attributes!comment}
\item
  What happens to a factor when you modify its levels?

\begin{Shaded}
\begin{Highlighting}[]
\NormalTok{f1 <-}\StringTok{ }\KeywordTok{factor}\NormalTok{(letters)}
\KeywordTok{levels}\NormalTok{(f1) <-}\StringTok{ }\KeywordTok{rev}\NormalTok{(}\KeywordTok{levels}\NormalTok{(f1))}
\end{Highlighting}
\end{Shaded}
\item
  What does this code do? How do \texttt{f2} and \texttt{f3} differ from
  \texttt{f1}?

\begin{Shaded}
\begin{Highlighting}[]
\NormalTok{f2 <-}\StringTok{ }\KeywordTok{rev}\NormalTok{(}\KeywordTok{factor}\NormalTok{(letters))}

\NormalTok{f3 <-}\StringTok{ }\KeywordTok{factor}\NormalTok{(letters, }\DataTypeTok{levels =} \KeywordTok{rev}\NormalTok{(letters))}
\end{Highlighting}
\end{Shaded}
\end{enumerate}

\hyperdef{}{matrices-and-arrays}{\section{Matrices and
arrays}\label{matrices-and-arrays}}

Adding a \texttt{dim()} attribute to an atomic vector allows it to
behave like a multi-dimensional \textbf{array}. A special case of the
array is the \textbf{matrix}, which has two dimensions. Matrices are
used commonly as part of the mathematical machinery of statistics.
Arrays are much rarer, but worth being aware of. \index{arrays|(}
\index{matrices|see{arrays}}

Matrices and arrays are created with \texttt{matrix()} and
\texttt{array()}, or by using the assignment form of \texttt{dim()}:

\begin{Shaded}
\begin{Highlighting}[]
\CommentTok{# Two scalar arguments to specify rows and columns}
\NormalTok{a <-}\StringTok{ }\KeywordTok{matrix}\NormalTok{(}\DecValTok{1}\NormalTok{:}\DecValTok{6}\NormalTok{, }\DataTypeTok{ncol =} \DecValTok{3}\NormalTok{, }\DataTypeTok{nrow =} \DecValTok{2}\NormalTok{)}
\CommentTok{# One vector argument to describe all dimensions}
\NormalTok{b <-}\StringTok{ }\KeywordTok{array}\NormalTok{(}\DecValTok{1}\NormalTok{:}\DecValTok{12}\NormalTok{, }\KeywordTok{c}\NormalTok{(}\DecValTok{2}\NormalTok{, }\DecValTok{3}\NormalTok{, }\DecValTok{2}\NormalTok{))}

\CommentTok{# You can also modify an object in place by setting dim()}
\NormalTok{c <-}\StringTok{ }\DecValTok{1}\NormalTok{:}\DecValTok{6}
\KeywordTok{dim}\NormalTok{(c) <-}\StringTok{ }\KeywordTok{c}\NormalTok{(}\DecValTok{3}\NormalTok{, }\DecValTok{2}\NormalTok{)}
\NormalTok{c}
\CommentTok{#>      [,1] [,2]}
\CommentTok{#> [1,]    1    4}
\CommentTok{#> [2,]    2    5}
\CommentTok{#> [3,]    3    6}
\KeywordTok{dim}\NormalTok{(c) <-}\StringTok{ }\KeywordTok{c}\NormalTok{(}\DecValTok{2}\NormalTok{, }\DecValTok{3}\NormalTok{)}
\NormalTok{c}
\CommentTok{#>      [,1] [,2] [,3]}
\CommentTok{#> [1,]    1    3    5}
\CommentTok{#> [2,]    2    4    6}
\end{Highlighting}
\end{Shaded}

\texttt{length()} and \texttt{names()} have high-dimensional
generalisations:

\begin{itemize}
\item
  \texttt{length()} generalises to \texttt{nrow()} and \texttt{ncol()}
  for matrices, and \texttt{dim()} for arrays. \indexc{nrow()}
  \indexc{ncol()} \indexc{dim()}
\item
  \texttt{names()} generalises to \texttt{rownames()} and
  \texttt{colnames()} for matrices, and \texttt{dimnames()}, a list of
  character vectors, for arrays. \indexc{rownames()} \indexc{colnames()}
  \indexc{dimnames()}
\end{itemize}

\begin{Shaded}
\begin{Highlighting}[]
\KeywordTok{length}\NormalTok{(a)}
\CommentTok{#> [1] 6}
\KeywordTok{nrow}\NormalTok{(a)}
\CommentTok{#> [1] 2}
\KeywordTok{ncol}\NormalTok{(a)}
\CommentTok{#> [1] 3}
\KeywordTok{rownames}\NormalTok{(a) <-}\StringTok{ }\KeywordTok{c}\NormalTok{(}\StringTok{"A"}\NormalTok{, }\StringTok{"B"}\NormalTok{)}
\KeywordTok{colnames}\NormalTok{(a) <-}\StringTok{ }\KeywordTok{c}\NormalTok{(}\StringTok{"a"}\NormalTok{, }\StringTok{"b"}\NormalTok{, }\StringTok{"c"}\NormalTok{)}
\NormalTok{a}
\CommentTok{#>   a b c}
\CommentTok{#> A 1 3 5}
\CommentTok{#> B 2 4 6}

\KeywordTok{length}\NormalTok{(b)}
\CommentTok{#> [1] 12}
\KeywordTok{dim}\NormalTok{(b)}
\CommentTok{#> [1] 2 3 2}
\KeywordTok{dimnames}\NormalTok{(b) <-}\StringTok{ }\KeywordTok{list}\NormalTok{(}\KeywordTok{c}\NormalTok{(}\StringTok{"one"}\NormalTok{, }\StringTok{"two"}\NormalTok{), }\KeywordTok{c}\NormalTok{(}\StringTok{"a"}\NormalTok{, }\StringTok{"b"}\NormalTok{, }\StringTok{"c"}\NormalTok{), }\KeywordTok{c}\NormalTok{(}\StringTok{"A"}\NormalTok{, }\StringTok{"B"}\NormalTok{))}
\NormalTok{b}
\CommentTok{#> , , A}
\CommentTok{#> }
\CommentTok{#>     a b c}
\CommentTok{#> one 1 3 5}
\CommentTok{#> two 2 4 6}
\CommentTok{#> }
\CommentTok{#> , , B}
\CommentTok{#> }
\CommentTok{#>     a  b  c}
\CommentTok{#> one 7  9 11}
\CommentTok{#> two 8 10 12}
\end{Highlighting}
\end{Shaded}

\texttt{c()} generalises to \texttt{cbind()} and \texttt{rbind()} for
matrices, and to \texttt{abind()} (provided by the \texttt{abind}
package) for arrays. You can transpose a matrix with \texttt{t()}; the
generalised equivalent for arrays is \texttt{aperm()}. \indexc{cbind()}
\indexc{rbind()} \indexc{abind()} \indexc{aperm()}

You can test if an object is a matrix or array using
\texttt{is.matrix()} and \texttt{is.array()}, or by looking at the
length of the \texttt{dim()}. \texttt{as.matrix()} and
\texttt{as.array()} make it easy to turn an existing vector into a
matrix or array.

Vectors are not the only 1-dimensional data structure. You can have
matrices with a single row or single column, or arrays with a single
dimension. They may print similarly, but will behave differently. The
differences aren't too important, but it's useful to know they exist in
case you get strange output from a function (\texttt{tapply()} is a
frequent offender). As always, use \texttt{str()} to reveal the
differences. \index{arrays!1d}

\begin{Shaded}
\begin{Highlighting}[]
\KeywordTok{str}\NormalTok{(}\DecValTok{1}\NormalTok{:}\DecValTok{3}\NormalTok{)                   }\CommentTok{# 1d vector}
\CommentTok{#>  int [1:3] 1 2 3}
\KeywordTok{str}\NormalTok{(}\KeywordTok{matrix}\NormalTok{(}\DecValTok{1}\NormalTok{:}\DecValTok{3}\NormalTok{, }\DataTypeTok{ncol =} \DecValTok{1}\NormalTok{)) }\CommentTok{# column vector}
\CommentTok{#>  int [1:3, 1] 1 2 3}
\KeywordTok{str}\NormalTok{(}\KeywordTok{matrix}\NormalTok{(}\DecValTok{1}\NormalTok{:}\DecValTok{3}\NormalTok{, }\DataTypeTok{nrow =} \DecValTok{1}\NormalTok{)) }\CommentTok{# row vector}
\CommentTok{#>  int [1, 1:3] 1 2 3}
\KeywordTok{str}\NormalTok{(}\KeywordTok{array}\NormalTok{(}\DecValTok{1}\NormalTok{:}\DecValTok{3}\NormalTok{, }\DecValTok{3}\NormalTok{))         }\CommentTok{# "array" vector}
\CommentTok{#>  int [1:3(1d)] 1 2 3}
\end{Highlighting}
\end{Shaded}

While atomic vectors are most commonly turned into matrices, the
dimension attribute can also be set on lists to make list-matrices or
list-arrays: \index{arrays!list-arrays} \index{list-arrays}

\begin{Shaded}
\begin{Highlighting}[]
\NormalTok{l <-}\StringTok{ }\KeywordTok{list}\NormalTok{(}\DecValTok{1}\NormalTok{:}\DecValTok{3}\NormalTok{, }\StringTok{"a"}\NormalTok{, }\OtherTok{TRUE}\NormalTok{, }\FloatTok{1.0}\NormalTok{)}
\KeywordTok{dim}\NormalTok{(l) <-}\StringTok{ }\KeywordTok{c}\NormalTok{(}\DecValTok{2}\NormalTok{, }\DecValTok{2}\NormalTok{)}
\NormalTok{l}
\CommentTok{#>      [,1]      [,2]}
\CommentTok{#> [1,] Integer,3 TRUE}
\CommentTok{#> [2,] "a"       1}
\end{Highlighting}
\end{Shaded}

These are relatively esoteric data structures, but can be useful if you
want to arrange objects into a grid-like structure. For example, if
you're running models on a spatio-temporal grid, it might be natural to
preserve the grid structure by storing the models in a 3d array.
\index{arrays|)}

\subsection{Exercises}

\begin{enumerate}
\def\labelenumi{\arabic{enumi}.}
\item
  What does \texttt{dim()} return when applied to a vector?
\item
  If \texttt{is.matrix(x)} is \texttt{TRUE}, what will
  \texttt{is.array(x)} return?
\item
  How would you describe the following three objects? What makes them
  different to \texttt{1:5}?

\begin{Shaded}
\begin{Highlighting}[]
\NormalTok{x1 <-}\StringTok{ }\KeywordTok{array}\NormalTok{(}\DecValTok{1}\NormalTok{:}\DecValTok{5}\NormalTok{, }\KeywordTok{c}\NormalTok{(}\DecValTok{1}\NormalTok{, }\DecValTok{1}\NormalTok{, }\DecValTok{5}\NormalTok{))}
\NormalTok{x2 <-}\StringTok{ }\KeywordTok{array}\NormalTok{(}\DecValTok{1}\NormalTok{:}\DecValTok{5}\NormalTok{, }\KeywordTok{c}\NormalTok{(}\DecValTok{1}\NormalTok{, }\DecValTok{5}\NormalTok{, }\DecValTok{1}\NormalTok{))}
\NormalTok{x3 <-}\StringTok{ }\KeywordTok{array}\NormalTok{(}\DecValTok{1}\NormalTok{:}\DecValTok{5}\NormalTok{, }\KeywordTok{c}\NormalTok{(}\DecValTok{5}\NormalTok{, }\DecValTok{1}\NormalTok{, }\DecValTok{1}\NormalTok{))}
\end{Highlighting}
\end{Shaded}
\end{enumerate}

\hyperdef{}{data-frames}{\section{Data frames}\label{data-frames}}

A data frame is the most common way of storing data in R, and if
\href{http://vita.had.co.nz/papers/tidy-data.pdf}{used systematically}
makes data analysis easier. Under the hood, a data frame is a list of
equal-length vectors. This makes it a 2-dimensional structure, so it
shares properties of both the matrix and the list. This means that a
data frame has \texttt{names()}, \texttt{colnames()}, and
\texttt{rownames()}, although \texttt{names()} and \texttt{colnames()}
are the same thing. The \texttt{length()} of a data frame is the length
of the underlying list and so is the same as \texttt{ncol()};
\texttt{nrow()} gives the number of rows. \index{data frames|(}

As described in \hyperref[subsetting]{subsetting}, you can subset a data
frame like a 1d structure (where it behaves like a list), or a 2d
structure (where it behaves like a matrix).

\subsection{Creation}

You create a data frame using \texttt{data.frame()}, which takes named
vectors as input:

\begin{Shaded}
\begin{Highlighting}[]
\NormalTok{df <-}\StringTok{ }\KeywordTok{data.frame}\NormalTok{(}\DataTypeTok{x =} \DecValTok{1}\NormalTok{:}\DecValTok{3}\NormalTok{, }\DataTypeTok{y =} \KeywordTok{c}\NormalTok{(}\StringTok{"a"}\NormalTok{, }\StringTok{"b"}\NormalTok{, }\StringTok{"c"}\NormalTok{))}
\KeywordTok{str}\NormalTok{(df)}
\CommentTok{#> 'data.frame':    3 obs. of  2 variables:}
\CommentTok{#>  $ x: int  1 2 3}
\CommentTok{#>  $ y: Factor w/ 3 levels "a","b","c": 1 2 3}
\end{Highlighting}
\end{Shaded}

Beware \texttt{data.frame()}'s default behaviour which turns strings
into factors. Use \texttt{stringAsFactors = FALSE} to suppress this
behaviour: \indexc{stringsAsFactors}

\begin{Shaded}
\begin{Highlighting}[]
\NormalTok{df <-}\StringTok{ }\KeywordTok{data.frame}\NormalTok{(}
  \DataTypeTok{x =} \DecValTok{1}\NormalTok{:}\DecValTok{3}\NormalTok{,}
  \DataTypeTok{y =} \KeywordTok{c}\NormalTok{(}\StringTok{"a"}\NormalTok{, }\StringTok{"b"}\NormalTok{, }\StringTok{"c"}\NormalTok{),}
  \DataTypeTok{stringsAsFactors =} \OtherTok{FALSE}\NormalTok{)}
\KeywordTok{str}\NormalTok{(df)}
\CommentTok{#> 'data.frame':    3 obs. of  2 variables:}
\CommentTok{#>  $ x: int  1 2 3}
\CommentTok{#>  $ y: chr  "a" "b" "c"}
\end{Highlighting}
\end{Shaded}

\subsection{Testing and coercion}

Because a \texttt{data.frame} is an S3 class, its type reflects the
underlying vector used to build it: the list. To check if an object is a
data frame, use \texttt{class()} or test explicitly with
\texttt{is.data.frame()}:

\begin{Shaded}
\begin{Highlighting}[]
\KeywordTok{typeof}\NormalTok{(df)}
\CommentTok{#> [1] "list"}
\KeywordTok{class}\NormalTok{(df)}
\CommentTok{#> [1] "data.frame"}
\KeywordTok{is.data.frame}\NormalTok{(df)}
\CommentTok{#> [1] TRUE}
\end{Highlighting}
\end{Shaded}

You can coerce an object to a data frame with \texttt{as.data.frame()}:

\begin{itemize}
\item
  A vector will create a one-column data frame.
\item
  A list will create one column for each element; it's an error if
  they're not all the same length.
\item
  A matrix will create a data frame with the same number of columns and
  rows as the matrix.
\end{itemize}

\subsection{Combining data frames}

You can combine data frames using \texttt{cbind()} and \texttt{rbind()}:
\indexc{cbind()} \indexc{rbind()}

\begin{Shaded}
\begin{Highlighting}[]
\KeywordTok{cbind}\NormalTok{(df, }\KeywordTok{data.frame}\NormalTok{(}\DataTypeTok{z =} \DecValTok{3}\NormalTok{:}\DecValTok{1}\NormalTok{))}
\CommentTok{#>   x y z}
\CommentTok{#> 1 1 a 3}
\CommentTok{#> 2 2 b 2}
\CommentTok{#> 3 3 c 1}
\KeywordTok{rbind}\NormalTok{(df, }\KeywordTok{data.frame}\NormalTok{(}\DataTypeTok{x =} \DecValTok{10}\NormalTok{, }\DataTypeTok{y =} \StringTok{"z"}\NormalTok{))}
\CommentTok{#>    x y}
\CommentTok{#> 1  1 a}
\CommentTok{#> 2  2 b}
\CommentTok{#> 3  3 c}
\CommentTok{#> 4 10 z}
\end{Highlighting}
\end{Shaded}

When combining column-wise, the number of rows must match, but row names
are ignored. When combining row-wise, both the number and names of
columns must match. Use \texttt{plyr::rbind.fill()} to combine data
frames that don't have the same columns.

It's a common mistake to try and create a data frame by
\texttt{cbind()}ing vectors together. This doesn't work because
\texttt{cbind()} will create a matrix unless one of the arguments is
already a data frame. Instead use \texttt{data.frame()} directly:

\begin{Shaded}
\begin{Highlighting}[]
\NormalTok{bad <-}\StringTok{ }\KeywordTok{data.frame}\NormalTok{(}\KeywordTok{cbind}\NormalTok{(}\DataTypeTok{a =} \DecValTok{1}\NormalTok{:}\DecValTok{2}\NormalTok{, }\DataTypeTok{b =} \KeywordTok{c}\NormalTok{(}\StringTok{"a"}\NormalTok{, }\StringTok{"b"}\NormalTok{)))}
\KeywordTok{str}\NormalTok{(bad)}
\CommentTok{#> 'data.frame':    2 obs. of  2 variables:}
\CommentTok{#>  $ a: Factor w/ 2 levels "1","2": 1 2}
\CommentTok{#>  $ b: Factor w/ 2 levels "a","b": 1 2}
\NormalTok{good <-}\StringTok{ }\KeywordTok{data.frame}\NormalTok{(}\DataTypeTok{a =} \DecValTok{1}\NormalTok{:}\DecValTok{2}\NormalTok{, }\DataTypeTok{b =} \KeywordTok{c}\NormalTok{(}\StringTok{"a"}\NormalTok{, }\StringTok{"b"}\NormalTok{),}
  \DataTypeTok{stringsAsFactors =} \OtherTok{FALSE}\NormalTok{)}
\KeywordTok{str}\NormalTok{(good)}
\CommentTok{#> 'data.frame':    2 obs. of  2 variables:}
\CommentTok{#>  $ a: int  1 2}
\CommentTok{#>  $ b: chr  "a" "b"}
\end{Highlighting}
\end{Shaded}

The conversion rules for \texttt{cbind()} are complicated and best
avoided by ensuring all inputs are of the same type.

\subsection{Special columns}

Since a data frame is a list of vectors, it is possible for a data frame
to have a column that is a list: \index{data frames!list in column}

\begin{Shaded}
\begin{Highlighting}[]
\NormalTok{df <-}\StringTok{ }\KeywordTok{data.frame}\NormalTok{(}\DataTypeTok{x =} \DecValTok{1}\NormalTok{:}\DecValTok{3}\NormalTok{)}
\NormalTok{df$y <-}\StringTok{ }\KeywordTok{list}\NormalTok{(}\DecValTok{1}\NormalTok{:}\DecValTok{2}\NormalTok{, }\DecValTok{1}\NormalTok{:}\DecValTok{3}\NormalTok{, }\DecValTok{1}\NormalTok{:}\DecValTok{4}\NormalTok{)}
\NormalTok{df}
\CommentTok{#>   x          y}
\CommentTok{#> 1 1       1, 2}
\CommentTok{#> 2 2    1, 2, 3}
\CommentTok{#> 3 3 1, 2, 3, 4}
\end{Highlighting}
\end{Shaded}

However, when a list is given to \texttt{data.frame()}, it tries to put
each item of the list into its own column, so this fails:

\begin{Shaded}
\begin{Highlighting}[]
\KeywordTok{data.frame}\NormalTok{(}\DataTypeTok{x =} \DecValTok{1}\NormalTok{:}\DecValTok{3}\NormalTok{, }\DataTypeTok{y =} \KeywordTok{list}\NormalTok{(}\DecValTok{1}\NormalTok{:}\DecValTok{2}\NormalTok{, }\DecValTok{1}\NormalTok{:}\DecValTok{3}\NormalTok{, }\DecValTok{1}\NormalTok{:}\DecValTok{4}\NormalTok{))}
\CommentTok{#> Error in data.frame(1:2, 1:3, 1:4, check.names = FALSE, stringsAsFactors = TRUE): arguments imply differing number of rows: 2, 3, 4}
\end{Highlighting}
\end{Shaded}

A workaround is to use \texttt{I()}, which causes \texttt{data.frame()}
to treat the list as one unit:

\begin{Shaded}
\begin{Highlighting}[]
\NormalTok{dfl <-}\StringTok{ }\KeywordTok{data.frame}\NormalTok{(}\DataTypeTok{x =} \DecValTok{1}\NormalTok{:}\DecValTok{3}\NormalTok{, }\DataTypeTok{y =} \KeywordTok{I}\NormalTok{(}\KeywordTok{list}\NormalTok{(}\DecValTok{1}\NormalTok{:}\DecValTok{2}\NormalTok{, }\DecValTok{1}\NormalTok{:}\DecValTok{3}\NormalTok{, }\DecValTok{1}\NormalTok{:}\DecValTok{4}\NormalTok{)))}
\KeywordTok{str}\NormalTok{(dfl)}
\CommentTok{#> 'data.frame':    3 obs. of  2 variables:}
\CommentTok{#>  $ x: int  1 2 3}
\CommentTok{#>  $ y:List of 3}
\CommentTok{#>   ..$ : int  1 2}
\CommentTok{#>   ..$ : int  1 2 3}
\CommentTok{#>   ..$ : int  1 2 3 4}
\CommentTok{#>   ..- attr(*, "class")= chr "AsIs"}
\NormalTok{dfl[}\DecValTok{2}\NormalTok{, }\StringTok{"y"}\NormalTok{]}
\CommentTok{#> [[1]]}
\CommentTok{#> [1] 1 2 3}
\end{Highlighting}
\end{Shaded}

\texttt{I()} adds the \texttt{AsIs} class to its input, but this can
usually be safely ignored. \indexc{I()}

Similarly, it's also possible to have a column of a data frame that's a
matrix or array, as long as the number of rows matches the data frame:
\index{data frames!array in column}

\begin{Shaded}
\begin{Highlighting}[]
\NormalTok{dfm <-}\StringTok{ }\KeywordTok{data.frame}\NormalTok{(}\DataTypeTok{x =} \DecValTok{1}\NormalTok{:}\DecValTok{3}\NormalTok{, }\DataTypeTok{y =} \KeywordTok{I}\NormalTok{(}\KeywordTok{matrix}\NormalTok{(}\DecValTok{1}\NormalTok{:}\DecValTok{9}\NormalTok{, }\DataTypeTok{nrow =} \DecValTok{3}\NormalTok{)))}
\KeywordTok{str}\NormalTok{(dfm)}
\CommentTok{#> 'data.frame':    3 obs. of  2 variables:}
\CommentTok{#>  $ x: int  1 2 3}
\CommentTok{#>  $ y: 'AsIs' int [1:3, 1:3] 1 2 3 4 5 6 7 8 9}
\NormalTok{dfm[}\DecValTok{2}\NormalTok{, }\StringTok{"y"}\NormalTok{]}
\CommentTok{#>      [,1] [,2] [,3]}
\CommentTok{#> [1,]    2    5    8}
\end{Highlighting}
\end{Shaded}

Use list and array columns with caution: many functions that work with
data frames assume that all columns are atomic vectors.
\index{data frames|)}

\subsection{Exercises}

\begin{enumerate}
\def\labelenumi{\arabic{enumi}.}
\item
  What attributes does a data frame possess?
\item
  What does \texttt{as.matrix()} do when applied to a data frame with
  columns of different types?
\item
  Can you have a data frame with 0 rows? What about 0 columns?
\end{enumerate}

\hyperdef{}{data-structure-answers}{\section{Answers}\label{data-structure-answers}}

\begin{enumerate}
\def\labelenumi{\arabic{enumi}.}
\item
  The three properties of a vector are type, length, and attributes.
\item
  The four common types of atomic vector are logical, integer, double
  (sometimes called numeric), and character. The two rarer types are
  complex and raw.
\item
  Attributes allow you to associate arbitrary additional metadata to any
  object. You can get and set individual attributes with
  \texttt{attr(x, "y")} and \texttt{attr(x, "y") \textless{}- value}; or
  get and set all attributes at once with \texttt{attributes()}.
\item
  The elements of a list can be any type (even a list); the elements of
  an atomic vector are all of the same type. Similarly, every element of
  a matrix must be the same type; in a data frame, the different columns
  can have different types.
\item
  You can make ``list-array'' by assuming dimensions to a list. You can
  make a matrix a column of a data frame with
  \texttt{df\$x \textless{}- matrix()}, or using \texttt{I()} when
  creating a new data frame \texttt{data.frame(x = I(matrix()))}.
\end{enumerate}
